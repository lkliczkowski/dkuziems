% kompilowac pdflatex-em
\documentclass[a4paper,11pt]{article}
\usepackage{polski}
\usepackage[utf8]{inputenc}
\usepackage[OT4]{fontenc}
\usepackage{graphicx}		% wstawianie, manipulowanie grafika
\usepackage{geometry}		% do ustawienia marginesow
\usepackage{url}		% obsluga url
\usepackage{hyperref}\hypersetup{colorlinks=true,urlcolor=blue}		%
\usepackage{multirow}
\usepackage{color}


% marginesy {lewy,,prawy}
\geometry{hdivide={1.75cm,,2.0cm}}
% marginesy {gora,,dol}
\geometry{vdivide={2.3cm,,2.3cm}}

% ustawienie odstępów między wierszami na 1.5
\renewcommand\baselinestretch{1.5}

% \hline\hline da linię pogrubioną
\setlength{\doublerulesep}{\arrayrulewidth}

\pagestyle{empty}

\hypersetup{
  pdftitle={Dariusz Piotr Kuziemski - curriculum vitae},
  pdfauthor={Dariusz Kuziemski},
}

\makeatletter
\def\section{\@ifstar\unnumberedsection\numberedsection}
\def\numberedsection{\@ifnextchar[%]
  \numberedsectionwithtwoarguments\numberedsectionwithoneargument}
\def\unnumberedsection{\@ifnextchar[%]
  \unnumberedsectionwithtwoarguments\unnumberedsectionwithoneargument}
\def\numberedsectionwithoneargument#1{\numberedsectionwithtwoarguments[#1]{#1}}
\def\unnumberedsectionwithoneargument#1{\unnumberedsectionwithtwoarguments[#1]{#1}}
\def\numberedsectionwithtwoarguments[#1]#2{%
  \ifhmode\par\fi
  \removelastskip
  \vskip 5ex\goodbreak
  \refstepcounter{section}%
  \hbox to \hsize{\hss\vbox{\advance\hsize by 0cm
      \noindent
      \leavevmode\Large\bfseries\raggedright
      \thesection.\ 
      #2\par
      \vskip -2ex
      \noindent
      \textcolor[rgb]{0.5,0.5,0.5}{\rule{\textwidth}{0.5pt}}
      }}\nobreak
  \vskip 2ex\nobreak
  \addcontentsline{toc}{section}{%
    \protect\numberline{\thesection}%
    #1}%
  }
\def\unnumberedsectionwithtwoarguments[#1]#2{%
  \ifhmode\par\fi
  \removelastskip
  \vskip 5ex\goodbreak
  \refstepcounter{section}% spis treści
  \hbox to \hsize{\hss\vbox{\advance\hsize by 0cm
      \noindent
      \leavevmode\Large\bfseries\raggedright\sc
%      \thesection\ 
      #2\par
      \vskip -4ex
      \noindent
%\hrulefill
\textcolor[rgb]{0.5,0.5,0.5}{\rule{\textwidth}{0.5pt}}
      }}\nobreak
  \vskip 2ex\nobreak
  \addcontentsline{toc}{section}{%
%    \protect\numberline{\thesection}%
    #1}%
  }

\makeatother
\pagestyle{empty}

\newcommand{\poz}[2]{
{\textsc{#1}} & #2\\}

\newcommand{\cvsection}[1]{\section*{#1}}

\newenvironment{cvtabular}%
	{\begin{tabular}{p{0.25\textwidth} p{0.75\textwidth}} }%
	{\end{tabular}}%

\newenvironment{cvheader}%
	{\begin{tabular}{p{0.22\textwidth} p{0.55\textwidth} p{0.15\textwidth}}%
	}%
	{\end{tabular}}%


\begin{document}
\centering{\Large{\textsc{Curriculum Vitae}}\\}
\centering{\LARGE{\textsc{Dariusz Kuziemski}}}

\cvsection{Dane osobowe}
	\begin{cvheader}
	\textsc{adres} & Semlin 39, 83--206 Kleszczewo & \multirow{5}{*}{ }\\
	\textsc{data urodzenia} & 21 grudnia 1987 \\
	\textsc{stan cywilny} & single \\
	\textsc{telefon} & 781044000 \\
	\textsc{email} &\href{mailto:kuziemski.dariusz@mailplus.pl}
			      {kuziemski.dariusz@mailplus.pl} \\
	\end{cvheader}
	

\cvsection{Wykształcenie}
  \begin{cvtabular}

      \poz{Paź 2009--do dziś}{Uniwersytet Gdański, Wydział MFI,
		      Informatyka/Inteligencja obliczeniowa, \newline
	      \begin{small}\emph{aktualnie piszę pracę magisterską
		      ``Algorytmy BI w Microsoft SQL Server 2008``}          
	      \end{small}\vspace*{0.2cm}}

      \poz{Paź 2008--do dziś}{Uniwersytet Gdański, Wydział MFI, Matematyka
		      \newline specjalizacja: Matematyka Ekonomiczna
	      \begin{small}\emph{(roczny urlop dziekański)}          
	      \end{small}\vspace*{0.2cm}}

      \poz{Lipiec 2009}{Uniwersytet Gdański, Wydział MFI, Informatyka 1
			stopnia, \newline \emph{Egzamin licencjacki}
			\vspace*{0.2cm}}

      \poz{Maj 2006}{I Liceum ogólnokształcące im. Marii Skłodowskiej-Curie
		    w Starogardzie \newline Gdańskim, \emph{Matura: j. polski,
		    j. angielski, Matematyka, Informatyka}}
  \end{cvtabular}

\cvsection{Nabyte umiejętności}
  \begin{cvtabular}
      \poz{języki \mbox{programowania}}{\textit{[Dobra znajomość]} C, C$\#$;

      \textit{[Podstawowa znajomość]} Ajax, ASP.NET, Assembler, bash, C++, CSS,
      \newline Fortran, Haskell, HTML, Java/J2EE, JavaScript, LaTeX, MDX, PHP,
      \newline Prolog, Python, SQL, XML \vspace*{0.35cm}}

      \poz{środowiska \mbox{programistyczne}}{Borland C++ Builder,
      Code::Blocks, Eclipse IDE, Microsoft Visual\newline Studio/SQL Server
      Business Intelligence Development Studio, MonoDevelop \vspace*{0.35cm}}

      \poz{systemy \mbox{operacyjne}}{Windows, GNU/Linux\vspace*{0.35cm}}

      \poz{inne}{Android SDK, Grafika 2D/3D (GIMP, Blender 3D), JADE,
      MVC(ASP.NET), Pakiety biurowe (typu Office), PostgreSQL, SVN, UML, Vim
      \vspace*{0.35cm}}

      \poz{obecnie \mbox{uczę się}}{C$\#$, Python}
  \end{cvtabular}

\pagebreak

\cvsection{Języki obce}
	\begin{cvtabular}
	\poz{angielski}{komunikatywny podstawowy, brak problemów ze
zrozumieniem dokumentacji}
	\poz{niemiecki}{podstawowy}
\end{cvtabular}

\cvsection{Doświadczenie}

\begin{cvtabular}
\begin{Large}Zawodowe\newline\end{Large}
 \poz{\href{http://it-portal.pl/}{it-portal.pl}}
  {\vspace*{0.5cm}Analityk IT, Paź 2009 - Lut 2010 \newline 
    \emph{ W ramach praktyk (matematyka ekonomiczna).\newline 
	  Jako analityk portalu byłem odpowiedzialny za wprowadzanie
	  skategoryzowanych informacji na temat oprogramowania biznesowego. } }
\end{cvtabular}

\begin{cvtabular}
\begin{Large}Projekty zespołowe\end{Large}
  \poz{\href{http://code.google.com/p/goodfeeling/}{GoodFeeling}}
  {\vspace*{0.5cm}Android SDK, Weka 3.6.3, Java \newline
    \emph{ Efektem była aplikacja data-mining na system Android. } }

 \poz{\href{http://code.google.com/p/padaly/source/browse/\#svn\%2Ftrunk\%2Fsrc}
     {jabsRoom}}{C, GTK+2.0, Empathy \newline 
    \emph{Efektem był komunikator w GTK opierający działanie na Empathy.} }
\end{cvtabular}

\begin{cvtabular}
\begin{Large}Własne\newline\end{Large}
  \poz{\href{http://code.google.com/p/dkuziems/source/browse/\#svn\%2Ftrunk}
      {svn-dkuziems}}{\vspace*{0.5cm}C, C$\#$ \newline 
    \emph{Ostatnio zacząłem prowadzić system kontroli wersji także dla projektów
  \newline własnych (np.: z sieci neuronowych.) } }
\end{cvtabular}

\cvsection{Dodatkowe umiejętności}
  \begin{cvtabular}
    \poz{Prawo Jazdy kat. B}{}
  \end{cvtabular}

\cvsection{Zainteresowania}
	\begin{cvtabular}
	\poz{naukowe}{data mining, sieci neuronowe}
	\poz{grafika 3D}{Blender 3D}
	\poz{literatura fantasy}{Forgotten Realms}
	\poz{muzyka}{Blues, Classic Rock (\emph{uczę się także grać na
gitarze})}
	\poz{sport}{piłka nożna, tenis stołowy}
\end{cvtabular}

\vspace{10ex}
\begin{center}
\textsf{\textcolor[rgb]{0.3,0.3,0.3}{
{\footnotesize Wyrażam zgodę na przetwarzanie moich danych osobowych dla
	    potrzeb niezbędnych do realizacji procesu rekrutacji\newline
	    (zgodnie z Ustawą z dn. 29.08.97 r. o Ochronie Danych Osobowych 
	    Dz. Ust. 133 poz. 883).}}}
\end{center}

\end{document}
